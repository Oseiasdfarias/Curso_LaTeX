\documentclass{article}

\usepackage[utf8]{inputenc} % Este pacote serve para acentuação
\usepackage[brazil]{babel} % Este pacote coloca os nomes em pt-br
\usepackage{indentfirst} % Este pacote aplica indentação
\usepackage[a4paper, left=1cm, right=2cm, top=2cm, bottom=3cm]{geometry} % Este pacote altera a margem do documento
\usepackage{graphicx} % Este pacote permite adicionar figuras
\usepackage{float} % Força o posicionamento
\usepackage{multirow} % Mesclar linhas
\usepackage{tabularx} % Margem da tabela
\usepackage{amsmath} % Modo matemático

\renewcommand{\sin}{\mathrm{sen\hspace{0.5mm}}}
\renewcommand{\tan}{\mathrm{tg\hspace{0.5mm}}}
\begin{document}	
	\title{\textbf{{\Huge Modo matemático - matrizes}}} % Título
	\author{Oséias Farias} % Autor
	\date{} % Data
	\maketitle % Criar o título, autor e data
	\thispagestyle{empty} % Oculta a numeração da página
	\newpage
	
	\setcounter{page}{1} % Começa a contar as páginas novamente
	\pagenumbering{Roman} % altera para algarismo romano
	\tableofcontents % Cria o sumário
	\newpage
	
	\listoffigures % Lista de figuras
	\newpage
	
	\listoftables % Lista de tabelas
	\newpage

	\setcounter{page}{1} % Começa a contar as páginas novamente
	\pagenumbering{arabic} % altera para algarismo arábico
	
	\section{Modo matemático}

    Esta é a equação de segundo grau: $ ax^2 +bx +c=0 $. A solução é:
    \begin{equation*}
    	x = \frac{-b \pm \sqrt{b^2 -4a\cdot c}}{2a}
    \end{equation*}
    
    \begin{equation*}
    	\begin{array}{cc}
    	x_1 = \dfrac{-b + \sqrt{b^2 -4a\cdot c}}{2a}, & 
    	x_2 = \dfrac{-b - \sqrt{b^2 -4a\cdot c}}{2a}
    	\end{array}	
    \end{equation*}
    
    \begin{equation*}
    	A = \begin{matrix}
    	1 & 0 & 0  \\
    	0 & 1 & 0  \\
    	0 & 0 & 1
    	\end{matrix}    	
    \end{equation*}
    
    \begin{equation*}
	    A = \begin{bmatrix}
	    1 & 0 & 0  \\
	    0 & 1 & 0  \\
	    0 & 0 & 1
	    \end{bmatrix}    	
    \end{equation*}
    
    \begin{equation*}
	    A = \begin{pmatrix}
	    1 & 0 & 0  \\
	    0 & 1 & 0  \\
	    0 & 0 & 1
	    \end{pmatrix}    	
    \end{equation*}
    
    \begin{equation*}
	    A = \begin{vmatrix}
	    1 & 0 & 0  \\
	    0 & 1 & 0  \\
	    0 & 0 & 1
	    \end{vmatrix}    	
    \end{equation*}
    
    \begin{equation*}
	    A = \begin{Vmatrix}
	    1 & 0 & 0  \\
	    0 & 1 & 0  \\
	    0 & 0 & 1
	    \end{Vmatrix}    	
    \end{equation*}
    
    \section{Redefinindo comandos seno e tangente}
    
    Vamos redefinir as funções matemáticas do inglês para o português\\
    
   \textbf{ Equação seno}
    \begin{equation*}
    	\sin(2x)
    \end{equation*}

    \textbf{Equação Tangente}
    \begin{equation*}
    \tan(2x)
    \end{equation*}
    
    \textbf{Frações entre parênteses}
    \begin{equation*}
    	\left(\frac{2a}{3e}\right)'
    \end{equation*}
    
   \textbf{ Equação entre colchetes}
    \begin{equation*}
    	\{2a\}
    \end{equation*}
    
    \textbf{Porcentagem}
    \begin{equation*}
    100 \%
    \end{equation*}
    
    \textbf{Sub escrito }
	\begin{equation*}
	x_{12}
	\end{equation*}
	

\end{document}